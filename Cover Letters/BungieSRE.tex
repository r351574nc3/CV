%% start of file `template.tex'.
%% Copyright 2006-2013 Xavier Danaux (xdanaux@gmail.com).
%
% This work may be distributed and/or modified under the
% conditions of the LaTeX Project Public License version 1.3c,
% available at http://www.latex-project.org/lppl/.


\documentclass[11pt,a4paper,sans]{moderncv}        % possible options include font size ('10pt', '11pt' and '12pt'), paper size ('a4paper', 'letterpaper', 'a5paper', 'legalpaper', 'executivepaper' and 'landscape') and font family ('sans' and 'roman')

% moderncv themes
\moderncvstyle{classic}                            % style options are 'casual' (default), 'classic', 'oldstyle' and 'banking'
\moderncvcolor{green}                              % color options 'blue' (default), 'orange', 'green', 'red', 'purple', 'grey' and 'black'
%\renewcommand{\familydefault}{\sfdefault}         % to set the default font; use '\sfdefault' for the default sans serif font, '\rmdefault' for the default roman one, or any tex font name
%\nopagenumbers{}                                  % uncomment to suppress automatic page numbering for CVs longer than one page

% character encoding
\usepackage[utf8]{inputenc}                       % if you are not using xelatex ou lualatex, replace by the encoding you are using
%\usepackage{CJKutf8}                              % if you need to use CJK to typeset your resume in Chinese, Japanese or Korean

% adjust the page margins
\usepackage[scale=0.75]{geometry}
%\setlength{\hintscolumnwidth}{3cm}                % if you want to change the width of the column with the dates
%\setlength{\makecvtitlenamewidth}{10cm}           % for the 'classic' style, if you want to force the width allocated to your name and avoid line breaks. be careful though, the length is normally calculated to avoid any overlap with your personal info; use this at your own typographical risks...

% personal data
\name{Leo}{Przybylski}
\title{Full Stack and DevOps Engineeer}                               % optional, remove / comment the line if not wanted
\address{2450 N. Palo Hacha}{85745 Tucson}{United States of America}% optional, remove / comment the line if not wanted; the "postcode city" and and "country" arguments can be omitted or provided empty
\phone[mobile]{+1~(520)~440~1252}                   % optional, remove / comment the line if not wanted
%\phone[fixed]{+2~(345)~678~901}                    % optional, remove / comment the line if not wanted
%\phone[fax]{+3~(456)~789~012}                      % optional, remove / comment the line if not wanted
\email{r351574nc3@gmail.com}                               % optional, remove / comment the line if not wanted
\homepage{https://www.linkedin.com/in/r351574nc3/}                         % optional, remove / comment the line if not wanted
%\extrainfo{additional information}                 % optional, remove / comment the line if not wanted
%\photo[64pt][0.4pt]{picture}                       % optional, remove / comment the line if not wanted; '64pt' is the height the picture must be resized to, 0.4pt is the thickness of the frame around it (put it to 0pt for no frame) and 'picture' is the name of the picture file
%\quote{Some quote}                                 % optional, remove / comment the line if not wanted

% to show numerical labels in the bibliography (default is to show no labels); only useful if you make citations in your resume
%\makeatletter
%\renewcommand*{\bibliographyitemlabel}{\@biblabel{\arabic{enumiv}}}
%\makeatother
%\renewcommand*{\bibliographyitemlabel}{[\arabic{enumiv}]}% CONSIDER REPLACING THE ABOVE BY THIS

% bibliography with mutiple entries
%\usepackage{multibib}
%\newcites{book,misc}{{Books},{Others}}
%----------------------------------------------------------------------------------
%            content
%----------------------------------------------------------------------------------
\begin{document}
%-----       letter       ---------------------------------------------------------
% recipient data
\recipient{Bungie}{Bellevue, WA}
\opening{Dear Sir or Madam,}
\closing{Sincerely,}
\enclosure[Attached]{curriculum vit\ae{}}          % use an optional argument to use a string other than "Enclosure", or redefine \enclname
\makelettertitle

%PARAGRAPH ONE: State the reason for the letter, name the position or
%type of work you are applying for and identify the source from which
%you learned of the opening.\\

I am applying for the position of Site Reliability Engineer at Bungie, Inc. I first learned of this opportunity through the Bungie, Inc. website. 
As a long-time gamer, I have followed Bungie, Inc. since Marathon. Even now, I play approximately an hour a day on Destiny 2. It has long been an 
ambition of mine to have the opportunity to work at Bungie, Inc. When I saw the description for the Gameplay Systems Engineer position, I immediately
knew that I had the skills and experience necessary for the position.

I am interested in applying over twenty years of full-stack software development experience as well as my resource management experience toward this position. My experience as a software developer and my inclination toward solving problems for developers has transformed me into someone that can build tools and processes that help developers and by strengthening their productivity. I am able to easily find possible points of failure and discover areas where human error can be replaced by better automation.

I have experience developing tools for IDEs, continuous integration and continuous delivery, deployment, architecture as code, pipeline as code, and real-time virtual machine provisioning. 

I am confident my qualifications meet the needs for Site Reliability Engineer at Bungie, Inc. I look forward to your response. My phone number is 520-440-1252.
I appreciate you for taking the time to consider my credentials.


\makeletterclosing

\end{document}


%% end of file `template.tex'.
